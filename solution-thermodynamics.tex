\documentclass[DIV15,11pt,parskip=half]{scrartcl}
\usepackage{amsmath}
\usepackage[T1]{fontenc}
\usepackage{natbib}
\usepackage{listings}
\newcommand{\changefont}[3]{\fontfamily{#1} \fontseries{#2} \fontshape{#3} \selectfont}

%% commands to facilitate units and temperature
\newcommand{\unit}[1]{\ensuremath{\,\mathrm{#1}}}
\newcommand{\s}[1]{\ensuremath{\,\mathrm{#1}}}
\newcommand{\cels}[1]{\ensuremath{#1^{\circ}\,\mathrm{C}}}

\usepackage{tikz}
\usetikzlibrary{shapes,arrows,decorations.pathmorphing,calc}

% ---------------------------------------------------------------------------------------
% definition of header and footer
% ---------------------------------------------------------------------------------------

\usepackage[automark,headsepline,footsepline]{scrpage2}
\clearscrheadings	
\ihead{Dynamics and thermodynamics of glaciers}
\ohead{McCarthy Summer School 2016}
\cfoot{\pagemark}
\setkomafont{pagehead}{\normalfont}	
\setkomafont{pagenumber}{\normalfont\rmfamily}




% ---------------------------------------------------------------------------------------
% Koma-Script - Settings
% ---------------------------------------------------------------------------------------
\addtokomafont{caption}{\small}
\setkomafont{captionlabel}{\sffamily\bfseries}
% \setkomafont{caption}{\sffamily}
\setcapindent{1em}

\pagestyle{scrheadings}


\begin{document}

\vspace{-5em}

\title{Thermodynamics of Glaciers \\[.2em]
\rule[1em]{\textwidth}{2pt}
\LARGE\textsf{Solutions to Exercise}
}
\date{}

\maketitle


\vspace{-5em}


\section{Climate history}

The initial amplitude is $\Delta T_0 = \cels{2}$ during an oscillation period of 50 years, thus $\omega = \frac{2\pi}{50\,\text{a}} = 3.9821\times 10^{-9}\,\text{s}^{-1}$. With Equation~(11) we get:

 \begin{equation*}
   \label{eq:temp-depth-amplitude}
   \frac{\Delta T(h)}{\Delta T_0}= \exp\left(-h \sqrt{\frac{\omega}{2\kappa}}\right).
 \end{equation*} The thermal diffusivity can be obtained from Equations~(2) and (3)
\begin{equation*}
\kappa(-\cels{3)} = \frac{k(-\cels{3})}{\rho\,c(-\cels{3})} = \frac{2.1\,\text{W}\,\text{m}^{-1}\,\text{K}^{-1}}{910\,\text{kg}\,\text{m}^{-3} \cdot 2105.7\,\text{J}\,\text{kg}^{-1}\,\text{K}^{-1}} = 1.34\times 10^{-6}\,\text{m}^{2}\,\text{s}^{-1}, 
\end{equation*} or, just use the value given in the script, $1.09\cdot 10^{-6}\unit{m}^{2}\unit{s}^{-1}$ (the result then may slightly vary).
Therefore we get
\begin{equation*}
  h = -\sqrt{\frac{2\kappa}{\omega}}\,\text{ln}\left(\frac{\Delta T(h)}{\Delta T_0}\right)
  = - \sqrt{\frac{2\cdot 1.34\times 10^{-6}\,\text{m}^{2}\,\text{s}^{-1}}{3.9821\times 10^{-9}\,\text{s}^{-1}}}\,\text{ln}\left( \frac{0.02}{2}\right) \approx 119\,\text{m}.
\end{equation*}

\section{Cold content}
 
The cold content can be obtained by
\begin{equation*}
E = -\int \limits_0^z \rho c_p T(z) \,\mathrm{d} z'
\end{equation*}
This can be approximated by the area under the depth profile:
\begin{equation*}
E \approx \frac{1}{2}\rho c_p\Delta T\Delta z = 105 \cdot 10^5\,\mathrm{J\,m}^{-2},
\end{equation*} where $\rho=500\,$kg\,m$^{-3}$ and $c_p=2009$\,J\,kg$^{-1}$\,K$^{-1}$.
Divide by the latent heat of fusion $L=3.34\cdot10^5\,\mathrm{J\,kg}^{-1}$ and you get 32\,kg\,m$^{-2}$ = 32 mm w.e..


\section{Melting temperature depression}

From Figure~(9) we see that the pressure melting temperature is $T_{\text{m}} = -\cels{0.34}$. We have $p = 910\,\text{kg}\,\text{m}^{3} \cdot 9.81\,\text{m}\,\text{s}^{-2}\cdot 300\,\text{m} + 75'000\,\text{Pa} = 2753130\,\text{Pa}$. Using Equation~(14) we obtain
  \begin{equation*}
    \gamma = -\frac{T_{\text{m}}-T_{\text{tp}}}{p - p_{\text{tp}}} = \frac{273.15\,\text{K} -0.34\,\text{K} -273.16\,\text{K}}{2753130\,\text{Pa} - 611.73\,\text{Pa}} = 1.202\times 10^{-7}\,\text{K}\,\text{Pa}^{-1}.
  \end{equation*} In the literature we find the values $\gamma = 7.42 \times 10^{-8} \,\text{K}\,\text{Pa}^{-1}$ for pure ice and air-free water and $\gamma = 9.8 \times 10^{-7} \,\text{K}\,\text{Pa}^{-1}$ for pure ice and air-saturated water. The value calculated for Gornergletscher is even higher than $\gamma$ for air-saturated water which means we have ice with air-saturated water.


\section{Lake Vostok}

\begin{enumerate}
\item Advection and diffusion
\item The P\'eclet number Pe is a measure of the relative importance of advection and diffusion.
\item First we calculate the pressure melting point at the base
\begin{align*}
 \label{eq:clausius-pure}
 T_m &= T_{tp} - \gamma\, (p - p_{\text{tp}}) \\
 & = 273.16\,\text{K} -  7.42 \times 10^{-8} \,\text{K}\,\text{Pa}^{-1} \left(  910\,\text{kg}\,\text{m}^{3} \cdot 9.81\,\text{m}\,\text{s}^{-2}\cdot 3300\,\text{m}- 611.73\,\text{Pa}\right) \\
 & \approx 271\,\text{K}
\end{align*} Then use Equation~(17) from the script:
\begin{equation*}
  T(z) = T_s + \frac{\sqrt\pi}{2} l \left(\frac{dT}{dz}\right)_{B} 
 \left[ {\textrm{erf}} \left( \frac{z}{l} \right) - {\textrm{erf}} \left( \frac{H}{l} \right) \right]\,.
\end{equation*} In order for a lake to form we need $T(0\,\text{m}) = T_{\text{m}}$.

\begin{align*}
  T(0\,\text{m}) & = T_s + \frac{\sqrt\pi}{2} l \left(\frac{dT}{dz}\right)_{B} 
 \left[ {\textrm{erf}} \left( \frac{z}{l} \right) - {\textrm{erf}} \left( \frac{H}{l} \right) \right] \\
 & = T_s + \frac{\sqrt\pi}{2} l \left(\frac{G}{k}\right) 
 \left[ {\textrm{erf}} \left( \frac{z}{l} \right) - {\textrm{erf}} \left( \frac{H}{l} \right) \right]
\end{align*}
The geothermal flux at the base thus must be equal or larger:
\begin{align*}
 G & \ge \frac{T_s - T(0\,\text{m})}{\frac{\sqrt\pi}{2} l \left(\frac{1}{k}\right) 
 \left[  - {\textrm{erf}} \left( \frac{H}{l} \right) \right]} \approx 0.05\,\text{W}\,\text{m}^{2}
\end{align*}


\end{enumerate}

\end{document}