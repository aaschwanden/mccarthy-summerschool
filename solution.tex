\documentclass[parskip=half]{scrartcl}
\usepackage{amsmath}
\usepackage[T1]{fontenc}
\usepackage{natbib}

\newcommand{\changefont}[3]{\fontfamily{#1} \fontseries{#2} \fontshape{#3} \selectfont}

%% commands to facilitate units and temperature
\newcommand{\unit}[1]{\ensuremath{\,\mathrm{#1}}}
\newcommand{\s}[1]{\ensuremath{\,\mathrm{#1}}}
\newcommand{\cels}[1]{\ensuremath{#1^{\circ}\,\mathrm{C}}}

\usepackage{tikz}
\usetikzlibrary{shapes,arrows,decorations.pathmorphing,calc}

% ---------------------------------------------------------------------------------------
% definition of header and footer
% ---------------------------------------------------------------------------------------

\usepackage[automark,headsepline,footsepline]{scrpage2}
\clearscrheadings	
\ihead{Dynamics and thermodynamics}
\ohead{McCarthy Summer School 2012}
\cfoot{\pagemark}
\setkomafont{pagehead}{\normalfont}	
\setkomafont{pagenumber}{\normalfont\rmfamily}




% ---------------------------------------------------------------------------------------
% Koma-Script - Settings
% ---------------------------------------------------------------------------------------
\addtokomafont{caption}{\small}
\setkomafont{captionlabel}{\sffamily\bfseries}
% \setkomafont{caption}{\sffamily}
\setcapindent{1em}

\pagestyle{scrheadings}


\begin{document}

\vspace{-5em}

\title{Dynamics and thermodynamics \\ of Glaciers \\[.2em]
\rule[1em]{\textwidth}{2pt}
\LARGE{\sf Solutions to Exercise}
}
\date{}

\maketitle


\vspace{-5em}

\section{Surface evolution}

\begin{enumerate}
\item In simple terms the surface responds to changes in dynamics (flux divergence) and climate (accumulation-ablation function)
\item
\begin{itemize}
\item in the accumulation zone, the surface rises 1\,m from September to March, and lowers by 1.5\,m from March to September. At the end of the hydrological year, the surface is thus 0.5\,m lower.
\item in the ablation zone, the surface rises 1\,m from September to March, and lowers by 1\,m from March to September. At the end of the hydrological year, the surface is thus the same.
\end{itemize}

% \begin{tikzpicture}[>=stealth,scale=1]
%       \coordinate (origin) at (0,0);
%       \coordinate (Tstart) at (-0.2,0);
%       \coordinate (Tend) at (9,0);
%       \coordinate (aplus) at +(0,2);
%       \coordinate (aminus) at +(0,-2);
%       \coordinate (start) at (0.1,0);
%       \coordinate (a2) at (4.1,1);
%       \coordinate (a3) at (8.1,-.5);
%       \coordinate (b2) at (4.1,1);
%       \coordinate (b3) at (8.1,0);
%    \draw[->] (Tstart) -- (Tend) node[anchor=north] {$t$};
%     \draw[->] (origin) -- (aplus) node[anchor=west] {$h$};
%     \draw[->] (origin) -- (aminus);
%     \draw[line width=1.5pt, color=blue] (start) -- (a2) -- (a3);
%     \draw[line width=1.5pt, color=red] (start) -- (b2) -- (b3);
% \end{tikzpicture}

\item because the glacier is shielded from ablation and accumulation (i.e. $a_{\textrm{s}} =0$), the glaciers slowly things and spreads under its own weight.
\end{enumerate}


\section{Mass balance, surface evolution, and vertical velocity}

Well, the example was taken slightly out of context, as the paper deals with rock glaciers. In this special case, the vertical velocity is indeed equal to the change of elevation with time. The term ``spatial mass balance'' is, however, an unlucky choice. Laser altimetry measures the surface elevation.


\section{Melting temperature depression}

From Figure~(9) we see that the pressure melting temperature is $T_{\text{m}} = -\cels{0.34}$. We have $p = 910\,\text{kg}\,\text{m}^{3} \cdot 9.81\,\text{m}\,\text{s}^{-2}\cdot 300\,\text{m} + 75'000\,\text{Pa} = 2753130\,\text{Pa}$. Using Equation~(14) we obtain
  \begin{equation*}
    \gamma = -\frac{T_{\text{m}}-T_{\text{tp}}}{p - p_{\text{tp}}} = \frac{273.15\,\text{K} -0.34\,\text{K} -273.16\,\text{K}}{2753130\,\text{Pa} - 611.73\,\text{Pa}} = 1.202\times 10^{-7}\,\text{K}\,\text{Pa}^{-1}.
  \end{equation*} In the literature we find the values $\gamma = 7.42 \times 10^{-8} \,\text{K}\,\text{Pa}^{-1}$ for pure ice and air-free water and $\gamma = 9.8 \times 10^{-7} \,\text{K}\,\text{Pa}^{-1}$ for pure ice and air-saturated water. The value calculated for Gornergletscher is even higher than $\gamma$ for air-saturated water which means we have ice with air-saturated water.


\section{Lake Vostok}

\begin{enumerate}
\item Advection and diffusion
\item The P\'eclet number Pe is a measure of the relative importance of advection and diffusion.
\item First we calculate the pressure melting point at the base
\begin{align*}
 \label{eq:clausius-pure}
 T_m &= T_{tp} - \gamma\, (p - p_{\text{tp}}) \\
 & = 273.16\,\text{K} -  7.42 \times 10^{-8} \,\text{K}\,\text{Pa}^{-1} \left(  910\,\text{kg}\,\text{m}^{3} \cdot 9.81\,\text{m}\,\text{s}^{-2}\cdot 3300\,\text{m}- 611.73\,\text{Pa}\right) \\
 & \approx 271\,\text{K}
\end{align*} Then use Equation~(21) from the script:
\begin{equation*}
  T(z) = T_s + \frac{\sqrt\pi}{2} l \left(\frac{dT}{dz}\right)_{B} 
 \left[ {\textrm{erf}} \left( \frac{z}{l} \right) - {\textrm{erf}} \left( \frac{H}{l} \right) \right]\,.
\end{equation*} In order for a lake to form we need $T(0\,\text{m}) = T_{\text{m}}$.

\begin{align*}
  T(0\,\text{m}) & = T_s + \frac{\sqrt\pi}{2} l \left(\frac{dT}{dz}\right)_{B} 
 \left[ {\textrm{erf}} \left( \frac{z}{l} \right) - {\textrm{erf}} \left( \frac{H}{l} \right) \right] \\
 & = T_s + \frac{\sqrt\pi}{2} l \left(\frac{G}{k}\right) 
 \left[ {\textrm{erf}} \left( \frac{z}{l} \right) - {\textrm{erf}} \left( \frac{H}{l} \right) \right]
\end{align*}
The geothermal flux at the base thus must be equal or larger:
\begin{align*}
 G & \ge \frac{T_s - T(0\,\text{m})}{\frac{\sqrt\pi}{2} l \left(\frac{1}{k}\right) 
 \left[  - {\textrm{erf}} \left( \frac{H}{l} \right) \right]} \approx 0.05\,\text{W}\,\text{m}^{2}
\end{align*}


\end{enumerate}

\end{document}